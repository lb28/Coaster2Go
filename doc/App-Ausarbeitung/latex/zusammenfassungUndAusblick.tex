\chapter{Ausblick}
\label{cha:Ausblick}

Wie im Kapitel \ref{cha:anforderungsabgleich} zu sehen ist, haben wir all unsere inhaltlichen Anforderungen an die App erfüllt. Deswegen haben wir uns in diesem Kapitel einige Gedanken gemacht, welche Funktionen  für eine zukünftige Erweiterung der App denkbar wären. Diese könnten die bisherigen Funktionen sinnvoll ergänzen.

\begin{itemize}
\item Wartezeitenverifizierung: Bisher gibt es in unserer App keine Möglichkeit, falsch eingetragene Wartezeiten zu melden oder rückgängig zu machen. Zwar gleichen sich einzelne falsch eingetragene Wartezeiten dank der verwendeten Metriken zur Berechnung der Durchschnittswartezeiten schnell aus und haben keinen großen Einfluss. Trotzdem wäre es eine gute Ergänzung, wenn andere Nutzer die zuletzt eingetragenen Wartezeiten bestätigen oder ablehnen könnten, sodass die jeweilige Wartezeit an Gewichtung zu- oder abnimmt.
\item Ansteh-Timer: Um die Wartezeit nicht jedes mal selbst messen oder ablesen zu müssen, wäre ein Ansteh-Timer eine gute Ergänzung für die Benutzer der App. Dazu wird beim Beginn des Anstehens ein Startknopf gedrückt, welcher den Timer startet. Kurz vor der Fahrt oder direkt nach der Fahrt kann dann der Stoppknopf gedrückt werden. Dadurch wird die gewartete Zeit automatisch ermittelt und hochgeldaen.
\item Bewertungskategorisierung: Bisher gibt es bei unseren Bewertungen für die Parks und Attraktionen jeweils nur die Möglichkeit, allgemein zwischen null und fünf Sternen zu verteilen. Noch besser wäre, für jeden Park oder jede Attraktion bestimmte Kriterien der Kategorisierung vorzunehmen. So könnten zum Beispiel Achterbahnen nach der Action, Wasserbahnen nach ihrer Erfrischung, Familienbahnen nach der Tauglichkeit für Kinder oder Restaurants nach dem Geschmack bewertet werden. Dies bietet sich besonders bei Attraktionen an, da sie zum einen bereits in verschiedene Kategorien aufgeteilt sind. Zum anderen laufen die Benutzer so nicht in Gefahr, zum Beispiel eine Kinderbahn niedriger zu bewerten, weil sie nicht genügend Action für sie bereit hält.
\item Tourplaner: Da die Attraktionen schon in einer Mapview angezeigt werden, wäre es eine sinnvolle Ergänzung, den Benutzern die Möglichkeit zu geben, sich selbst eine Tour zusammen zu stellen, welche sie dann auf der Karte angezeigt bekommen. Desweiteren könnten auch bereits vorhandene Touren geladen werden oder Touren automatisch erstellt werden, zum Beispiel nach Kategorien oder nach Wartezeiten.
\end{itemize}

Zudem besteht zumindest theoretisch die Möglichkeit, Freizeitparks und ihre Daten direkt mit in die App einzubinden. Dadurch ergibt sich auch die Möglichkeit, wie sich die App finanzieren könnte. Freizeitparks, die eine Kooperation eingehen, könnten Vorteile oder Premiumaccounts erhalten. Mögliche Vorteile wären neben dem selbstständigen Verwalten der eigenen Parkdaten zum Beispiel das Einbinden ihrer eigenen Live-Wartezeiten aus bereits vorhandenen parkeigenen APIs, das Schalten von Werbung oder Aktionen des Freizeitparks oder auch das Hervorheben des Freizeitparks durch ein besonderes Design oder eine gut sichtbare Platzierung in der Übersichtsliste. 

\chapter{Fazit}
\label{cha:Fazit}

In diesem Kapitel soll ein kurzes Fazit zur App-Entwicklung und zum Anwendungsfach allgemein festgehalten werden.

Dadurch, dass wir als Team schon im Semester zuvor die Quartett-App zusammen entwickelt haben, hatten wir schon genügend Vorwissen, um in die Entwicklung unserer Freizeitpark-App einzusteigen. Trotzdem haben wir auch während der Entwicklung dieser App viel Neues gelernt was wir für die Zukunft und für die Entwicklung weiterer Apps gebrauchen können, wie zum Beispiel das Arbeiten mit Standorten und das Einbinden von vorhandenen Frameworks.

In unseren Augen ist uns die Coaster2Go App größtenteils gut gelungen und das Endergebnis ist genau so, wie wir es uns am Anfang des Semesters vorgestellt haben. Die App wäre theoretisch schon ausreichend für eine Veröffentlichung im App Store, da es bisher keine vergleichbare App im Google Play Store gibt. Natürlich gäbe es noch einige Stellen, die eine Anpassung benötigen, bevor es wirklich zu einer Veröffentlichung kommen kann. Zum Beispiel müsste die Speicherung der Daten auf einen anderen Server verlegt werden, da wir bisher mit unseren Studentenaccounts nur begrenzte Kapazitäten besitzen. Auch kleinere gestalttechnische oder funktionale Verbesserungen sind denkbar. 

Auch was das Arbeiten im Team und die Aufteilung der Aufgaben angeht, konnten wir im Vergleich zum ersten Teil des Anwendungsfaches noch Neues dazu lernen. Die Teamarbeit hat immer gut funktioniert und es kam nie zu größeren Komplikationen. Wir werden es auf jeden Fall in Erwägung ziehen, möglicherweise im Masterstudium das Anwendungsfach fortzusetzen.
