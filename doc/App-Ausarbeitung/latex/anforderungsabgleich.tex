% tabellenzeile dreispaltig
\newcommand{\td}[3]{
	#1 & #2 & #3 \\ \hline
}

\chapter{Anforderungsabgleich}
\label{cha:anforderungsabgleich}

Es folgt ein tabellarischer Vergleich zwischen den in Kapitel \ref{cha:anforderungsanalyse} 
spezifizierten Anforderungen und den implementierten Funktionen. Es wird für jede der Anforderungen 
beurteilt, inwieweit diese in der Implementierung erfüllt ist (Ja / Teilweise / Nein). Eine 
Zuordnung zu den Einträgen aus Kapitel \ref{cha:anforderungsanalyse} ist anhand der ID möglich.

Diejenigen Funktionen, die implementiert wurden, jedoch nicht im Voraus als Anforderungen 
spezifiziert waren, werden in diesem Vergleich nicht berücksichtigt.

\clearpage

% Abschnitt: Funktionale Anforderungen
\section{Funktionale Anforderungen}
\label{sec:anforderungsabgleich:funktional}

% padding anpassen für tabelle
\setlength{\tabcolsep}{10pt}
\renewcommand{\arraystretch}{1.3}

\begin{center}
	\begin{table}[h]
		\begin{tabular}{ | l | l | p{11.5cm} |}
			%TODO Breite automatisch die Seite füllen lassen? (\textwidth ist zu lang)
			\hline
			\textbf{ID} & \textbf{Erfüllt} & \textbf{Kommentar} \\ \hline
			\td{FA1}{Nein}{Unnötig, da die Account-Funktionen im Sidebar-Menü untergebracht wurden 
			und die Anwendung direkt auf der Parkübersicht startet.}
			\td{FA2}{Ja}{-}
			\td{FA3}{Ja}{-}
			\td{FA4}{Ja}{-}
			\td{FA5}{Ja}{-}
			\td{FA6}{Ja}{-}
			\td{FA7}{Ja}{-}
			\td{FA8}{Ja}{Es sind weiterhin 4 verschiedene Statistiken, jedoch wurde die Semantik 
			leicht geändert, um sie hilfreicher zu machen (siehe Kapitel 4.2.1 
			``Wartezeit-Berechnung und -darstellung'').}
			\td{FA9}{Ja}{-}
			\td{FA10}{Ja}{-}
		\end{tabular}
	\end{table}
\end{center}

% padding wieder auf standard
\setlength{\tabcolsep}{6pt}
\renewcommand{\arraystretch}{1}

% Abschnitt: Nichtfunktionale Anforderungen
\section{Nichtfunktionale Anforderungen}
\label{sec:anforderungsabgleich:nichtfunktional}

\begin{center}
	\begin{tabular}{ | l | l | p{11.5cm} |}
		\hline
		\textbf{ID} & \textbf{Erfüllt} & \textbf{Kommentar} \\ \hline
		\td{NFA1}{Ja}{-}
		\td{NFA2}{Ja}{Die Zeit zum Hoch- oder Herunterladen von Daten liegt oft weit über 1,5 
		Sekunden, aber dieser Vorgang blockiert nicht die Benutzeroberfläche und kann abgebrochen 
		werden.}
		\td{NFA3}{Ja}{Die Anforderung ist nur schwer zu beurteilen. Es wurde sich möglichst an die 
		Android-Richtlinien für Benutzeroberflächen gehalten, und alle Bereiche der 
		Anwendung sind nach dem Empfinden aller bisherigen Benutzer leicht bedienbar.}
		\td{NFA4}{Ja}{-}
	\end{tabular}
\end{center}
