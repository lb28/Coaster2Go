\chapter{Einleitung}
\label{cha:einleitung}

In dieser Dokumentation wird die Entwicklung einer App im Rahmen des Anwendungsfaches Mobile Application Development an der Universität Ulm und die dadurch erstellte App in Form einer Freizeitpark-App vorgestellt.

% Abschnitt: Problemstellung
\section{Motivation und Problemstellung}
\label{sec:einleitung:problemstellung}

Die Problemstellung war für dieses Projekt wesentlich offener gegeben, da wir eine beliebige App für unsere präferierte mobile Platform entwickeln sollten. Nachdem wir einige Ideen gesammelt hatten, uns Gedanken zur Implementierung gemacht haben und die Vor- und Nachteile abgewägt hatten, entschieden wir und für eine Freizeitpark-App. Als zentrale Platform für Friezeitparkliebhaber sollten Parks mit Standorten, zugehörige Attraktionen mit Wartezeiten und Bewertungen verfügbar sein. 

Eine große Anzahl potentieller Nutzer von Freizeitpark-Apps dürfte vorhanden sein. So hat zum Beispeil alleine der Europapark, Deutschlands größter Freizeitpark, jährlich 5,5 Millionen Besucher, Tendenz steigend\footnote{Quelle: \url{ http://presse.europapark.com/de/presse/nachricht/datum/2017/07/21/europa-park-ist-tourismusmagnet-im-schwarzwald/}}. Zusammen mit allen anderen Freizeitparks in Deutschland oder sogar weltweit kommt schnell eine beachtliche Summe von Freizeitparkbesuchern und damit potentiellen Nutzern zusammen, die wir mit unserer App ansprechen wollen. Umso erstaunlicher ist es, dass bisher nicht wirklich Apps oder Dienste für Freizeitparkbesucher zur Verfügung stehen.

Wenn man einmal die aktuelle Marktlage betrachtet fällt auf, dass es nur Apps für einzelne bestimmte große Parks, wie zum Beispiel Disneyworld, eigens entwickelte Apps oder von Fans entwickelte Apps gibt. Bei genauerer Betrachtung zeigen diese Apps aber Mängel auf. Sie bieten kaum oder nur wenig Funktionen, sind teilweise recht veraltet und nicht benutzerfreundlich designt. Wir konnten keine App für Android finden, welche im Ansatz der unserer Vorstellung ähnelte.

\begin{figure}[htp]
	\centering
  	\includegraphics[width=0.75\textwidth]{img/motivation/bluefire.jpg}
	\caption[Blue Fire Achterbahn]{Symbolbild: Potentielle Nutzer\footnotemark}
	\label{figure:bluefire}
\end{figure}
\footnotetext{Bildquelle: \url{https://de.wikipedia.org/wiki/Datei:Europa-Park_-_Blue_Fire_Megacoaster_(32).JPG}}

% Abschnitt: Zielsetzung
\section{Zielsetzung}
\label{sec:einleitung:zielsetzung}

Da also auf dem Markt eine Nachfrage nach einer Freizeitpark-App entsteht, wollen wir diese ausnutzen um im Laufe dieses Projekts eine eigene Anwendung zu entwickeln. Diese wollen wir auf Basis von Android und Java implementieren. Dabei geht es uns primär darum, den Umgang mit den neuen Techniken (beispielsweise GPS) zu erlernen und unsere bereits vorhandenen Kenntnisse zu erweitern. 

Inhaltlich möchten wir eine Anwendung entwickeln in der es möglich sein sollte aus einer Liste von Freizeitparks, deren Attraktionen anzusehen. Inklusive zu den Attraktionen bieten wir die zugehörigen Wartezeiten, Statistiken und Bewertungen. 
Ebenfalls sollte es die App dem Benutzer ermöglichen selbst Wartezeiten einzutragen und Parks bzw. die Attraktionen zu bewerten.
Außerdem sollen in einem Admin-Bereich die Funktionen zur Erstellung und Bearbeitung eigener Parks und Attraktionen vorhanden sein.
Des Weiteren stehen auch die Benutzerfreundlichkeit, das Design und die Kompatibilität mit möglichst vielen Android-Smartphones im Vordergrund, sodass ein breites Spektrum an Nutzern angesprochen wird. 

% Abschnitt: Struktur der Arbeit
\section{Struktur der Arbeit}
\label{sec:einleitung:struktur}

In dieser Dokumentation werden zuerst einmal die Grundlagen der Bewertungen und Wartezeitenberechnung erklärt, sowie Android vorgestellt und unsere verwendeten Frameworks präsentiert. Den Hauptteil bildet die Implementierung, welche unsere App im Allgemeinen und mit ihren Besonderheiten vorstellt. Außerdem wird darin unsere Architektur gezeigt und wir erleutern Schwierigkeiten, die wir während der Implementierungsphase hatten. Abschließend gibt es einen Anforderungsabgleich und einen Ausblick auf die Zukunft des Projektes.\\

\begin{figure}[htp]
	\centering
  	\includegraphics[width=0.4\textwidth]{img/coaster42_logo.png}
	\caption{Coaster2go Logo}
	\label{figure:coaster2gologo}
\end{figure}
