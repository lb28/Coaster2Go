\newcommand{\reqtable}[3]{
	\vspace{-1cm}
    \begin{center}
    \begin{tabular}{ | l | p{13cm} |}
    \hline
    \textbf{ID:} & \textbf{#1} \\ \hline
    TITEL: & #2 \\ \hline
    BES: & #3 \\
    \hline
    \end{tabular}
    \end{center}
}

\chapter{Anforderungsanalyse}	
\label{cha:anforderungsanalyse}
In diesem Kapitel werden die ursprünglich definierten funktionalen und nichtfunktionalen Anforderungen tabellarisch dargestellt. Jede der Anforderungen hat einen eindeutigen Identifikator (ID), einen Titel (TITEL), und eine Beschreibung (BES).

% Abschnitt: Funktionale Anforderungen
\section{Funktionale Anforderungen}
\label{sec:anforderungsanalyse:funktional}

\reqtable{FA1}{Startbildschirm}{Nach dem Start der Anwendung sieht der Benutzer einen 
Startbildschirm, auf dem er sich anmelden bzw. registrieren kann, oder ohne Anmeldung zur 
Parkübersicht gelangt.}

\reqtable{FA2}{Parkübersicht}{Diese Ansicht zeigt eine Liste aller Parks, die auf verschiedene 
Weisen sortiert werden kann. Außerdem gibt es eine getrennte Liste, in der sich die favorisierten 
Parks des Nutzers befinden. Sofern die Position des Smartphones bekannt ist, wird die Entfernung 
neben den Parks angezeigt und die Liste kann nach Entfernung sortiert werden.}

\reqtable{FA3}{Park Detailansicht}{Für jeden Park gibt es eine Detailansicht mit näheren 
Informationen. Hier befinden sich Links/Buttons zu den Bewertungen, der Kartenansicht, sowie zur 
Attraktionsübersicht des Parks. Außerdem kann der Park zu den Favoriten hinzugefügt werden.}

\reqtable{FA4}{Bewertungsseite (Park / Attraktionen)}{Es gibt eine Ansicht der neuesten Bewertungen 
(nach Datum sortiert). Eine Bewertung hat jeweils Verfasser, Datum, Stern-Anzahl, und Kommentar. 
Falls der Benutzer eingeloggt ist, sieht er einen Button zum Hinzufügen einer Bewertung.}

\reqtable{FA5}{Bewertung verfassen}{Sofern eingeloggt, kann der Benutzer in einem separaten Dialog 
eine Bewertung verfassen. Hat er zuvor schon eine Bewertung zu diesem Park / dieser Attraktion 
verfasst, kann er diese bearbeiten. Der Administrator kann den Kommentar ausblenden.}

\reqtable{FA6}{Attraktionsübersicht}{Diese Ansicht zeigt eine Liste aller Attraktionen, die auf 
verschiedene Weisen sortiert und gefiltert werden kann. Außerdem gibt es eine getrennte Liste, in 
der sich die favorisierten Attraktionen des Nutzers befinden. Neben jeder Attraktion wird, sofern 
verfügbar, die jeweilige „aktuelle“ Wartezeit angezeigt. (Sofern die Position des Smartphones 
bekannt ist, wird die Entfernung neben den Parks angezeigt und die Liste kann nach Entfernung 
sortiert werden.)}

\reqtable{FA7}{Attraktion Detailansicht}{Für jede Attraktion gibt es eine Detailansicht mit näheren 
Informationen. Hier befinden sich Links/Buttons zu den Bewertungen, der Kartenansicht, sowie zu den 
Wartezeiten. Auch kann die Attraktion zu den Favoriten hinzugefügt werden. Zudem gibt es 
verschiedene Statistiken zu den Wartezeiten (siehe FA8). Ist der Nutzer eingeloggt und befindet 
sich in der Nähe des Parks (GPS muss aktiviert sein), so kann er eine Wartezeit eintragen. Dies 
kann er nur ein Mal pro Stunde machen.}

\reqtable{FA8}{Wartezeit-Statistiken}{Es gibt vier verschiedene Statistiken über die Wartezeit bei 
einer Attraktion:
\begin{itemize}
	\item „Aktuell“: Zeigt die aktuelle Wartezeit an
	\item „Heute“: Zeigt den Durchschnitt für den aktuellen Tag an
	\item „Gesamt“: Zeigt den bisherigen Gesamtdurchschnitt an
	\item Ein Säulendiagramm, welches den stündlichen bisherigen Durchschnitt anzeigt
\end{itemize}
}

\reqtable{FA9}{Wartezeiten-Verlauf}{Es gibt eine Listenansicht der bisher eingetragenen Wartezeiten 
einer Attraktion. Ein Wartezeiten-Eintrag enthält Verfasser, Datum, Uhrzeit, und Wartezeit (in 
Minuten).}

\reqtable{FA10}{Admin-Funktionen}{Ein Administrator hat folgende Zusatzfunktionen:
\begin{itemize}
\item Parks / Attraktionen hinzufügen und erstellte bearbeiten
\item Kommentare ausblenden
\end{itemize}
}

% Abschnitt: Nichtfunktionale Anforderungen
\section{Nichtfunktionale Anforderungen}
\label{sec:anforderungsanalyse:nichtfunktional}

\reqtable{NFA1}{Entwicklungssprache und -Umgebung}{Die Anwendung wird in Java mit der 
Entwicklungsumgebung „Android Studio“ entwickelt und soll auf Geräten mit Betriebssystemen ab 
Android 4.1 funktionieren.}

\reqtable{NFA2}{Reaktionszeit}{Die Reaktionszeit der Anwendung soll zu jeder Zeit maximal 1,5 
Sekunden betragen}

\reqtable{NFA3}{Benutzbarkeit}{Die Anwendung soll intuitiv bedienbar sein. D.h., die Buttons und 
andere Auswahlmöglichkeiten sollen eine ausreichende Größe haben und wie erwartet reagieren.}

\reqtable{NFA4}{Offline-Modus}{Die Anwendung soll alle Anfragen zwischenspeichern, um auch Offline 
Zugriff auf die wichtigen Informationen zu gewährleisten (Bilder werden von der Bild-Bibliothek 
gecached).}